\documentclass{article}
\usepackage{graphicx} % Required for inserting images

\title{\textbf{Universidad Veracruzana} }
\date{\textbf{Facultad de Negocios y Tecnologias} }

\begin{document}
\maketitle
%\section
\textsf{\Large \textbf{Experiencia Educativa:} Paradigmas De Programación.\\}
 
\maketitle
%\section
\textsf{\Large \textbf{Alumno:} Gutierrezz Votte Carlos Ehytam. \\}

\maketitle
%\section
\textsf{\Large \textbf{Tema:} Reporte Técnico. \\}

\textsf{\ \textbf{Grupo:} 402 ISW 1° Parcial \\}

\maketitle

\textsf{\ Fecha de Entrega: 17 de Marzo del 2023 \\}

\author{GutierrezVotte}
\date{Marzo 2023}

\newpage

\maketitle
%\section{Integrantes}
\textsf{\ \\
\textbf{Introducción:}\\
\\
En el presente informe técnico, se discutirá sobre las redes neuronales Hopfield, su importancia y el impacto que tienen en diversos campos. Se centrará en el enfoque de matrices y el reconocimiento de patrones de vocales mayúsculas utilizando este método.
//El proceso de detección y clasificación de objetos a través de matrices binarias es sorprendente y útil en varias áreas de la vida cotidiana. En este informe, se relatará mi experiencia al trabajar con esta técnica y se presentará una conclusión personal sobre su relevancia y aplicabilidad en el mundo actual. \\}
\\
\textbf{Las Redes de Hopfield:}
\begin{itemize}
    \item Hopfield conceptualizó las redes neuronales como
        sistemas dinámicos con energía y mostró su
        semejanza con ciertos modelos físicos.

    \item Hopfield propuso varios modelos de redes
recurrentes. En este tipo de redes, la salida de cada
neurón se calcula y se retro-alimenta como entrada,
calculándose otra vez, hasta que se llega a un punto
de estabilidad. 
    \item Supuestamente los cambios en las salidas van
siendo cada vez mas pequeños, hasta llegar a cero,
esto es, alcanzar la estabilidad.
    \item Puede ser que una red recurrente nunca llegue a un
punto estable.   
\end{itemize}
\textsf{P. Gómez Gil. INAOE,(2017)}
\\


\textbf{Red:}
\begin{itemize}
    \item Principalmente con entradas binarias.
    \item Hopfield también utilizó sus redes para resolver un
problema de optimización: el agente viajero. Además
construyó una red con circuitos integrados que convierte
señales analógicas en digitales. 
\end{itemize}
\textsf{P. Gómez Gil. INAOE,(2017)}
\\
\newpage
\textbf{Matriz de representación:}
\begin{itemize}
    \item Asimismo, la matriz se puede representarse
en un vector con N2 elementos, donde N es
el número de ciudades.
    \item A su vez, este vector puede representarse en
una red de Hopfield de N2 neurones.
    \item El objetivo del entrenamiento es hacer
converger la red hacia una ruta válida, en el
cual exista la mínima energía posible.  
\end{itemize}
\textsf{P. Gómez Gil. INAOE,(2017)}

\\En el otro proyecto, se presenta una técnica brillante para el reconocimiento de vocales mayúsculas en matrices mediante el uso del método Hopfield. Es sorprendente ver cómo la tecnología puede identificar y clasificar patrones complejos en datos de entrada. El proceso de reconocimiento de objetos es fascinante y cada vez más importante en diversas áreas, desde la seguridad hasta la automatización industrial. Es emocionante explorar y aprender más sobre cómo funcionan estos algoritmos y cómo pueden aplicarse en la vida diaria.



\textbf{Las redes neuronales Hopfield tienen muchas aplicaciones
prácticas en la vida diaria:}
\begin{itemize}
    \item Reconocimiento de patrones: Las redes Hopfield se utilizan en el reconocimiento de patrones, como el reconocimiento de caracteres escritos a mano. Los modelos de redes Hopfield pueden aprender y almacenar patrones, lo que les permite identificar patrones similares en nuevos datos.
    \item Optimización: Las redes Hopfield también se utilizan en problemas de optimización, como la optimización de la distribución de energía o la optimización del enrutamiento de vehículos. Los modelos de redes Hopfield pueden encontrar soluciones óptimas para problemas de optimización.
    \item Modelado de sistemas biológicos: Las redes Hopfield se utilizan para modelar sistemas biológicos, como la memoria y la cognición. Los modelos de redes Hopfield pueden simular la forma en que el cerebro almacena y recupera información.
\end{itemize}
\textbf{Hopfield en simbolos}
\\





\textbf{Conclusión:}

Las redes Hopfield son ampliamente utilizadas en diversas áreas, como la optimización, el reconocimiento de patrones y el modelado de sistemas biológicos. Los ejemplos de referencia que se han mencionado previamente muestran la aplicación práctica y relevancia de las redes Hopfield en la actualidad. Estas redes son una herramienta poderosa que ha demostrado ser útil en una variedad de aplicaciones de la vida real



\textbf{Bibliografía:}
\begin{itemize}
    \item Alom, M. Z., Yakopcic, C., Hasan, M., Taha, T. M., & Asari, V. K. (2019). Recurrent Convolutional Neural Networks for Text Classification. Neural Computing and Applications, 31(2), 515-526.
    \item Gao, J., Zeng, Y., Yang, Z., Huang, Y., & Zhou, Y. (2020). Self-Attention Based Convolutional Neural Networks for Social Recommendation. Knowledge-Based Systems, 207, 106349.
  
\end{itemize}
\end{document}